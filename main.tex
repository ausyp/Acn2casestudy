
\documentclass[a4paper,man,british]{apa6}
\usepackage[british]{babel}
\usepackage[utf8]{inputenc}
\usepackage{csquotes}
\usepackage[hidelinks]{hyperref}
\usepackage[style=apa]{biblatex}
\DeclareLanguageMapping{british}{british-apa}

% maps apacite commands to biblatex commands
%\let \citeNP \cite
%\let \citeA \textcite
%\let \cite \parencite

% disclosure footnote

\newcommand\blfootnote[1]{%
  \begingroup
  \renewcommand\thefootnote{}\footnote{#1}%
  \addtocounter{footnote}{-1}%
  \endgroup
}

\addbibresource{zotero_references.bib}
\addbibresource{bibliography.bib}

\title{Case Study: A discussion of Post-operative and Pre-operative management of a patient undergoing total hip replacement for a fractured neck of femur}
\shorttitle{pre and post operative management of THR}
\author{Austin Paul}
\affiliation{RMIT UNIVERSITY \\ s3634517 \\ Word Count : 000}




\begin{document}

\maketitle

\section{}
\section{Introduction} %100
%
\blfootnote{*I had done this Case study before and had received instructor feedback on it.}
%state about pre and post-op care. dont refer
This case study attempts to discuss pre-operative and post-operative care required for an aged care resident who needs to under go total hip replacement due to a fall at home. In this instance the patient must have been presented to the emergency department post fall and escalated to the orthopaedics department from there. As the fracture would require immediate surgery there maybe significant limitation to pre-operative care. The discussion will include pre-operative assessment, patient education/information, surgery optimisation, addressing patient concerns, significance of multidisciplinary intervention in speeding up recovery, post-operative pain management and rehabilitation.



%%%***preop***%%

\section{Preoperative care} %650

According to \textcite{molko_rapid_2017} An effective pre-operative care strategy is essential for post-operative recovery, reducing the length of stay at hospital and rehabilitation. \citeauthor{molko_rapid_2017} also mentions Pre-operative care is a multi-modal and multidisciplinary endeavour which includes allied health professionals, surgeons anaesthetists, physiotherapists and nurses. In the circumstances  stated in the case study the patient has a fracture on the neck of femur after fall at home. The biggest concern for this patient would be pain management as patients admitted to ED with a hip fracture generally give a visual analogue scale pain level of eight out of ten \parencite{monzon_pain_2010}.
\newpage
\subsection{Pre-operative Pain Management}

As pain management can be considered as the prime concern for this patient. An effective pain management strategy can reduce the immediate discomfort due to pain associated with hip fracture but could aid in diagnostic investigation and early mobilisation in a post-operative setting \parencite{fernandez_management_2015,monzon_pain_2010}. \textcite{fernandez_management_2015} also suggests that nerve blocks should be the recommended option pain relief in case of elderly with hip fracture as it provides effective pain relief in the short term and helps minimise the the need for opioid analgesics and its consequent side effects like respiratory depression and opioid induced delirium.

\subsection{Pre-operative assessment and Investigation}

Pre-operative assessments are required to establish safety and fitness for surgery, this is important because geriatric conditions are predictors for adverse surgical outcome \parencite{oresanya_preoperative_2014}. \textcite{simunovic_effect_2010} has shown that there is a strong co-relation between early surgery and a low rates of mortality and pressure sore and pneumonia. Generally hip fractures are treated as an emergency and a corrective surgery is done with in forty eight hours of admission \parencite{siu_preoperative_2010}. An assessment of patients  patients fitness to receive anaesthetic and investigation like an echo cardiograph, in some rare circumstances may show that patient may have cardiac cardiac complications like an aortic stenosis and that a surgery would be an unacceptably high risk, in such circumstances it may be decided to manage the fracture with analgesics, this approach does come with cost of longer hospital stay and associated co morbidity's \parencite{parker_hip_2006,siu_preoperative_2010}. \textcite{oheireamhoin_role_2011} has shown that 51\% of patients who had preoperative cardiac investigation had their medication due to the findings. But it is also worth mentioning that these investigation did slow down the surgery and had to wait longer than patients who hadn't had these investigations, in cases of emergency surgeries there is a real risk of paralysis by analysis. 
Clinical guidelines by \textcite{chesser_new_2011} advises the use of a combination MRI X-ray and CT scan to establish the extend of fracture
\newline
Further, assessments should include the patients ability make an informed decision. Approaches like requesting the patient to describe the risks and nature of the proposed procedure can help in determining the extent of the patients understanding regarding the procedure \parencite{oresanya_preoperative_2014}.
A range of motion review should be under taken to estimate post-operative likely hood of dislocation and establish patient expectation \parencite{krenzel_high_2010}.

\subsection{Pre-operative planing and interventions}

Patients about to undergo surgery would have multiple prophylactic interventions in place. Patients would not usually have anticoagulant  or similar therapy due to high risk of bleeding but its up to discretion of the surgeon. The risk of infection is really high in post operative and therefore prophylactic IV antibiotics are usually administered \parencite{lundine_adherence_2010}.
A clear post-operative plan can help in reducing patients anxiety and help 

%%%***postop***%%


%https://www.ncbi.nlm.nih.gov/pmc/articles/PMC4133447/

\section{} %650




\newpage
\section{Conclusion}%100



\printbibliography

\end{document}
