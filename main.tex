
\documentclass[a4paper,man,british]{apa6}
\usepackage[british]{babel}
\usepackage[utf8]{inputenc}
\usepackage{csquotes}
\usepackage[hidelinks]{hyperref}
\usepackage[style=apa]{biblatex}
\DeclareLanguageMapping{british}{british-apa}

% maps apacite commands to biblatex commands
%\let \citeNP \cite
%\let \citeA \textcite
%\let \cite \parencite

% disclosure footnote

\newcommand\blfootnote[1]{%
  \begingroup
  \renewcommand\thefootnote{}\footnote{#1}%
  \addtocounter{footnote}{-1}%
  \endgroup
}

\addbibresource{zotero_references.bib}
\addbibresource{bibliography.bib}

\title{Case Study: A discussion of Post-operative and Pre-operative management of a patient undergoing total hip replacement for a fractured neck of femur}
\shorttitle{pre and post operative management of THR}
\author{Austin Paul}
\affiliation{RMIT UNIVERSITY \\ s3634517 \\ Word Count : 000}




\begin{document}

\maketitle

\section{}
\section{Introduction} %100
%
\blfootnote{*I had done this Case study before and had received instructor feedback on it.}
%state about pre and post-op care. dont refer
This case study attempts to discuss pre-operative and post-operative care required for an aged care resident who needs to under go total hip replacement due to a fall at home. In this instance the patient must have been presented to the emergency department post fall and escalated to the orthopaedics department from there. As the fracture would require immediate surgery there maybe significant limitation to pre-operative care. The discussion will include pre-operative assessment, patient education/information, surgery optimisation, addressing patient concerns, significance of multidisciplinary intervention in speeding up recovery, post-operative pain management and rehabilitation.



%%%***preop***%%

\section{Preoperative care} %650

According to \textcite{molko_rapid_2017} An effective pre-operative care strategy is essential for post-operative recovery, reducing the length of stay at hospital and rehabilitation. \citeauthor{molko_rapid_2017} also mentions Pre-operative care is a multi-modal and multidisciplinary endeavour which includes allied health professionals, surgeons anaesthetists, physiotherapists and nurses. In the circumstances  stated in the case study the patient has a fracture on the neck of femur after fall at home. The biggest concern for this patient would be pain management as patients admitted to ED with a hip fracture generally give a visual analogue scale pain level of eight out of ten \parencite{monzon_pain_2010}.
\newpage
\subsection{Pre-operative Pain Management}

As pain management can be considered as the prime concern for this patient. An effective pain management strategy can reduce the immediate discomfort due to pain associated with hip fracture but could aid in diagnostic investigation and early mobilization in a post-operative setting \parencite{fernandez_management_2015,monzon_pain_2010}. \textcite{fernandez_management_2015} also suggests that nerve blocks should be the recommended option pain relief in case of elderly with hip fracture as it provides effective pain relief in the short term and helps minimise the the need for opioid analgesics and its consequent side effects like respiratory depression and opioid induced delirium.
\subsection{Pre-operative assessment and establishing fitness for surgery}
The 



%%%***postop***%%


%https://www.ncbi.nlm.nih.gov/pmc/articles/PMC4133447/

\section{} %650




\newpage
\section{Conclusion}%100



\printbibliography

\end{document}
